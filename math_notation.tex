% !TEX TS-program = pdflatex
% !TEX encoding = UTF-8 Unicode

\documentclass[letterpaper, 11pt]{article}


% layout/spacing related packages
\usepackage[margin=1in]{geometry}
\usepackage{setspace}\onehalfspace

% Math related packages
\usepackage{amsmath} \allowdisplaybreaks
\usepackage{amssymb}
\usepackage{amsthm}



%%%%%%%%%%%%%% MACRO %%%%%%%%%%%%%%%%%%%%%%%%%%%%%%%%%%%%%%
\usepackage{bm}
\renewcommand{\vec}[1]{\boldsymbol{\mathbf{#1}}}
\newcommand{\mat}[1]{\vec{#1}}
\newcommand{\set}[1]{\mathcal{#1}}

% https://tex.stackexchange.com/questions/265689/ignore-greek-letters-using-mathcal
\DeclareMathSymbol{\Gamma}{\mathord}{operators}{"00}
\DeclareMathSymbol{\Delta}{\mathord}{operators}{"01}
\DeclareMathSymbol{\Theta}{\mathord}{operators}{"02}
\DeclareMathSymbol{\Lambda}{\mathord}{operators}{"03}
\DeclareMathSymbol{\Xi}{\mathord}{operators}{"04}
\DeclareMathSymbol{\Pi}{\mathord}{operators}{"05}
\DeclareMathSymbol{\Sigma}{\mathord}{operators}{"06}
\DeclareMathSymbol{\Upsilon}{\mathord}{operators}{"07}
\DeclareMathSymbol{\Phi}{\mathord}{operators}{"08}
\DeclareMathSymbol{\Psi}{\mathord}{operators}{"09}
\DeclareMathSymbol{\Omega}{\mathord}{operators}{"0A}

\newcommand{\LB}{\mathsf{LB}}
\newcommand{\UB}{\mathsf{UB}}		
			
\usepackage{etoolbox}
% To use, for example, \Ac for \mathcal{A}.  Requires "etoolbox" package
\makeatletter
\def\do#1{\@namedef{#1c}{\ensuremath{\mathcal{#1}}}}
\docsvlist{A,B,C,D,E,F,G,H,I,J,K,L,M,N,O,P,Q,R,S,T,U,V,W,X,Y,Z}
\makeatother

\def\Eb{\mathbb{E}}
\def\Rb{\mathbb{R}}

\newcommand{\dx}{\mathop{}\!\mathrm{d}x}
\newcommand{\dy}{\mathop{}\!\mathrm{d}y}
\newcommand{\dz}{\mathop{}\!\mathrm{d}z}
\newcommand{\dt}{\mathop{}\!\mathrm{d}t}

\renewcommand{\bar}[1]{\mkern 1.5mu\overline{\mkern-1.5mu#1\mkern-1.5mu}\mkern 1.5mu}



%%%%%%%%%%%%%%%%%%%%%%%%%%%%%%%%%%%%%%%%%%%%%%%%%%%%%%%%
% To show LaTeX code examples

\usepackage{fancyvrb,listings}
\lstset{
   breaklines=true,
   basicstyle=\linespread{0.8}\ttfamily\scriptsize,
   aboveskip=0pt,
   belowskip=0pt
}
\newenvironment{example}
 {\VerbatimOut{\jobname.tmp}}
 {\endVerbatimOut
 \begin{center}
 \fbox{
	 \begin{minipage}[c]{\textwidth}
	  \lstinputlisting{\jobname.tmp}
	 \end{minipage}
   }
 \fbox{
 	\begin{minipage}[c]{\textwidth}
	 \input{\jobname.tmp}
  	\end{minipage}
  }
  \end{center}
 }

%%%%%%%%%%%%%%%%%%%%%%%%%%%%%%%%%%%%%%%%%%%%%%%%%%%%%%%%





\title{Suggestions for the Mathematical Notation}
\author{}
\date{}


\begin{document}

\maketitle

A good notation helps easier reading. Consistency is the key. Try to follow these suggestions. 
\begin{example}
\begin{description}
\item [Sets:] $\set{B}_i$, $\set{K}$
\item [Vectors:] $\vec{x}$, $\vec{\xi}$
\item [Matrices:] $\mat{A}$
\item [Elements of vectors and matrices:] $x_j$, $\xi_j$, $a_{ij}$
\item [Textual names:] $Z_{\text{WDP}}$, $u^{\min}$, $\mathsf{UB}$
\item [Random variables:] $X$, $Y$
\item [Expected value:] $\Eb[X]$, $\Eb[g(Y)]$
\item [Set of real numbers:] $\Rb$, $\Rb^n$, $\Rb^n \times \Rb^m$, $\Rb^{n+m}$
\end{description}
\end{example}

Note that many macros are used. Check the preamble of this \texttt{.tex} file.

Some other suggestions:

\begin{enumerate}
\item Try to use the same alphabet for related concepts. For example:
	\begin{enumerate}
	\item A vector $\vec{x}$ belongs to set $\set{X}$. Similarly, $\vec{\gamma}\in\set{\Gamma}$
	\item The bound on variable $q_i$ is $Q_i$.
	\item Set of time periods: $\set{T} = \{1,2,...,T\}$ and each time period $t\in\set{T}$. The final time period is $T$. If you need a dummy index for time, consider $\tau$ or $s$:
		\begin{itemize}
		\item $\displaystyle x_t = \sum_{\tau=t}^T y_\tau$
		\item $\displaystyle y_t = \sum_{s=t}^T z_s$		
		\end{itemize}
	\item When $\phi(\cdot)$ is a function, its integral may be $\Phi(x) = \int_0^x \phi(y) \dy$. Similarly $F(x) = \int_0^x f(y) \dy$.
	\end{enumerate}
\item If you need to use bar/hat/tilde, try to keep the meaning of them consistent. For example:
	\begin{enumerate}
	\item If you use $\bar{x}$ to denote a solution obtained by CPLEX, then $\bar{y}$ should also be a solution obtained by CPLEX.
	\item If you use $\tilde{\vec{x}}$ to denote an approximation to vector $\vec{x}$, then $\tilde{\mat{A}}$ should also be an approximation to matrix $\mat{A}$ and $\tilde{f}(\vec{x})$ should be an approximation to function $f(\vec{x})$.
	\end{enumerate}
\item Use Roman alphabets for primal variables $x$, $y$, $z$ and Greek alphabets for dual variables $\xi$, $\gamma$, $\theta$.
\item Try to avoid text in your notation. If you have to, try the followings:
	\begin{enumerate}
	\item $Z_{\text{WDP}}$ instead of $Z_{WDP}$. Is $W$, $D$, and $P$ are separate indices for $Z$? Or does it mean $W \times D \times P$?
	\item $\text{(profit)} = \text{(revenue)} - \text{(cost)} $ instead of $profit = revenue - cost$. It looks $profit = p\times r\times o\times f\times i\times t$.
	\item If you want to define a textual variable name such as UB and LB for upper and lower bounds, for example, then try to use $\mathsf{UB}$ and $\mathsf{LB}$. While UB can be a \emph{generic} shorthand for the text ``upper bound'', $\mathsf{UB}$ is a mathematical symbol that has a \emph{specific} definition. You can use $\mathsf{UB}$ during the algorithm description; for example, ``Update as follows: $\mathsf{UB} \gets \min\{\mathsf{UB}, f(x^*) + g(x^*; \bar{y})\}$.''
		\begin{itemize}
		\item worst: The optimality gap is defined as $(UB - LB)/LB$.
		\item better: The optimality gap is defined as $(\text{UB} - \text{LB})/\text{LB}$.
		\item best: The optimality gap is defined as $(\mathsf{UB} - \mathsf{LB})/\mathsf{LB}$.
		\end{itemize}
		It may be useful to define macros
			\begin{verbatim}
			\newcommand{\LB}{\mathsf{LB}}
			\newcommand{\UB}{\mathsf{UB}}			
			\end{verbatim}
		 Then use as
		 \begin{verbatim}
		 (\UB - \LB) / \LB	
		 \frac{\UB - \LB}{\LB}
		 \end{verbatim}
		\item Similarly, CVaR is a \emph{generic} acronym for the text `conditional value-at-risk', and $\mathsf{CVaR}_\alpha$ is a \emph{specific} math symbol with $\alpha$ as a probability threshold.	 
	\end{enumerate}
\end{enumerate}


\end{document}







%
